\section{Simulation digitization}

\subsection{Description and features of libsbsdig}

The currently developped version of libsbsdig is a standalone program which uses the information from g4sbs to produce realistic ADC and TDC values from the detector.
Similarly to g4sbs, the output ADC and TDC values are stored as series of stl vectors
The necessary input to operate this program is fairly simple:
%
\begin{itemize}
\item{one unique configuration file to setup all the detectors tunable parameters (i.e. gains, pedestals means and widths, thresholds, number of ADC and TDC bits...}
\item{one text file to list the g4sbs input files.}
\item{optionally one input root file storing background histograms (see section~\ref{bkgdadd})}
\end{itemize}
%
A more complete documentation is available on\\
{https://redmine.jlab.org/projects/sbs-software/wiki/Documentation\_of\_libsbsdig}

\subsection{Digitization algorithms}
\label{digalgo}

In this section we detail the algorithms which are used to digitize

\subsubsection{GEMs}

The digitization algorithm for the GEMs can be decomposed into four main steps.
%
\paragraph{Step~1: Ionization}
During this step we perform the following actions:
%
\begin{itemize}
\item{Use the energy deposit to estimate the number of ions generated;}
\item{Distribute these ions uniformly between $x_{in}$, $y_{in}$, $z_{in}$ and $x_{out}$, $y_{out}$, $z_{out}$;}
\end{itemize}
%
\paragraph{Step~2: Avalanche}
%
In this step we simulate the avalanche for each of these ions, according to a Cauchy-Lorentz function (which width will depend on the speed of diffusion of the ions in the gas, etc, and the amplitude will depend on the gain);
%
\paragraph{Step~3: Numerical integration of the avalanche}
In this step we numerically integrate the avalanche on each of the strips: the integrated charge normalizes a pulse function $f(t) = C (t-t_0)/\tau^2 exp(-(t-t_0)/\tau)$. (where $\tau$ = 56ns and $t_0$ depends on the time of the hit registered by g4sbs)
This function is then integrated on the six 25ns samples and converted into ADC values.
%
\paragraph{Step~4: Pedestal addition and saturation}
The pedestal is added on top of the ADC value for each of the sample. The pedestal value is indicated in the configuration file, and the pedestal width is typically $\sigma$ = 20 ADC.
The total ADC value is then capped to the maximal ADC value which for the APV chips is $2^{12} = 4096$ to simulate ADC saturation.

Note: In the full background case, steps 1, 2, 3 are repeated for the background hits before applying the pedestal and the saturation.


\subsubsection{PMT detectors}

The digitization algorithm for the PMT detectos can be decomposed into four main steps. There is a bit of variability depending on the readout electronics (if it uses TDCs or not).

\paragraph{Step~1: Number of photoelectrons estimation}
The number of photoelectrons is from the energy deposit. This estimation depends on the detector e.g. for the timing hodoscope or the HCal, there is a dependence of the light yield on the hit position; The number of photoelectrons multiplied by the PMT gain is used to normalize a pulse function  $f(t) = C (t-t_0)/\sigma^2 exp(-(t-t_0)/\sigma)$ where $\sigma$ is the FWHM of the PMT provided in the configuration file.
%add in details

\paragraph{Step~2: Timing estimation for TDCs}
If the DAQ involves TDCs, the pulse is compared to the threshold (provided in the configuration file).
If the pulse crosses the threshold, the leading and 
trailing times are recorded.
%add in figure

\paragraph{Step~3: ADC/TDCs conversion}
The integral of the pulse is converted into ADC values; 
the times are converted into TDC values; In the case of HCal 
(readout by FADC 250), the pulse is integrated on 4ns ADC samples 

\paragraph{Step~4: }

4th step: apply a pedestal on the ADC value(s). The mean and width of the pedestal is provided in the configuration file;
The total ADC value is then capped to the maximal ADC value  $2^{ADC\_bits}$ (with $ADC\_bits$ provided in the configuration file) to simulate ADC saturation.

Note: In the full background case, steps 1, 2, 3 are repeated for the background hits before applying the pedestal and the saturation.

\subsection{Background addition}
\label{bkgdadd}

The first step for the background addition is a ``pre-processing'' of beam-on-target g4sbs files to generate background histograms for:
%
\begin{itemize}
\item{hit multiplicity (within a certain time window);}
\item{distribution of hit position;}
\item{energy deposit}
\item{number of photoelectrons (when applicable);}
\end{itemize}
%
These histograms are fed as an input to sbsdig, which uses them to (re)generate hits by sampling
those histograms.
These hits time is distributed uniformly within the detector time window.
Once those hits are generated, they are processed and superimposed to the signal hits (see section~\ref{digalgo}).
