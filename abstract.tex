\begin{abstract}
  Super Bigbite Spectrometer is a new instrument in preparation to take data in Hall~A at Jefferson Laboratory starting in 2021.
  It will consist of a large aperture magnet with a modular detector package, and will be combined together with another arm (that will vary depending on the measurement).
  Its core physics program consists in the measurement of the nucleon form factors at large values of $Q^2$, but it is versatile enough to perform other measurements such as semi-inclusive DIS or even tagged DIS.
  Those measurements have in common to require high luminosity, which, combined with the large solid angle and open geometry, induces large trigger and background rates, which makes those measurements particularly challenging.

  Overcoming those challenges will require a lot of preparation including simulations to both evaluate actual experimental conditions and prepare pseudodata samples to develop the analysis.
  
  This document reviews the simulation and software framework, with a specific focus on the simulation digitization and its interface to the SBS analysis software SBS-offline.
  %I will also illustrate the experimental challenge we have to face and overcome with the example of the tracking for the measurement of the proton electric form factor $G_E^p$.
\end{abstract}

{\bf keywords:} High luminosity/rates/background experiments, simulation, software.
